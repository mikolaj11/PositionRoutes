\documentclass[submission]{FPSAC2020}

\usepackage{cite}
\usepackage{lipsum}
\usepackage[utf8]{inputenc}
\usepackage[polish, english]{babel}
\usepackage{tikz}
\usepackage{pgfplots}
\usepackage{todonotes}
\usepackage{subfiles}
\usepackage[labelfont={normalsize}]{caption,subfig}
\usepackage{amsmath,amssymb, amsthm}
\usepackage{graphicx}
    
\newcommand*{\bigs}[1]{\vcenter{\hbox{\scalebox{3.5}{\ensuremath#1}}}}

\newcommand{\verker}
{
	\begin{tikzpicture}
	\begin{axis}[
		axis x line*=bottom,
  		axis y line*=left,
  		enlargelimits=false,
  		xmin=0,
		xmax=2.4,
		ymin=0,
		ymax=2.2
	]

	\addplot [
		domain=-2:2, 
		samples=1000, 
		color=black,
		style={ultra thick}
	]
	({1/pi*(x*rad(asin(x/2))+sqrt(4-x*x))+x/2},{1/pi*(x*rad(asin(x/2))+sqrt(4-x*x))-x/2});
	\addplot[
	only marks,
    color=black,
    mark=o,
	style={thick}
    ]
    coordinates {
    (0,2)
    };
    \addplot[
    color=black,
    mark=o,
	style={thick}
    ]
    coordinates {
    (2,0)
    };
	\fill[thick,draw=black,fill=white] ({100*(1/pi*(1.5*rad(asin(0.5/2))+sqrt(4-1.5*1.5))+0.5/2)},{100*(1/pi*(0.5*rad(asin(1.5/2))+sqrt(4-0.5*0.5))-0.5/2)}) circle (100*0.03) node[anchor=south] {\tiny \large \hspace{45pt}$G(0.6)$};
	\fill[thick,draw=black,fill=white] ({100*(1/pi*(0.5*rad(asin(1.5/2))+sqrt(4-0.5*0.5))-0.5/2)},{100*(1/pi*(1.5*rad(asin(0.5/2))+sqrt(4-1.5*1.5))+0.5/2)}) circle (100*0.03) node[anchor=south west] {\tiny \large \hspace{-10pt} $G(0.4)$};
	\fill[thick,draw=black,fill=white] ({100*(1/pi*(1*rad(asin(1/2))+sqrt(4-1*1))+1/2)},{100*(1/pi*(1*rad(asin(1/2))+sqrt(4-1*1))-1/2)}) circle (100*0.03) node[anchor=south] {\tiny \large \hspace{50pt}$G(0.8)$};
	\fill[thick,draw=black,fill=white] ({100*(1/pi*(1*rad(asin(1/2))+sqrt(4-1*1))-1/2)},{100*(1/pi*(1*rad(asin(1/2))+sqrt(4-1*1))+1/2)}) circle (100*0.03) node[anchor=south west] {\tiny \large \hspace{-8pt} $G(0.2)$};
	\fill[thick,draw=black,fill=white] ({100*(1/pi*(-2*rad(asin(-2/2))+sqrt(4-2*2))-2/2)},{100*(1/pi*(-2*rad(asin(-2/2))+sqrt(4-2*2))+2/2)}) circle (100*0.03) node[anchor=west] {\tiny \large \hspace{4pt}$G(0.0)$};
	\fill[thick,draw=black,fill=white] ({100*(1/pi*(2*rad(asin(2/2))+sqrt(4-2*2))+2/2)},{100*(1/pi*(2*rad(asin(2/2))+sqrt(4-2*2))-2/2)}) circle (100*0.03) node[anchor=south] {\tiny \large $G(1.0)$};

	\end{axis}
	\end{tikzpicture}
}


\newcommand{\trzy}
{
	\colorlet{CiemnoZielony}{green!50!black}
\subfloat[]{
\begin{tikzpicture}[scale=0.08]
\begin{scope}
\clip (0,0) rectangle (42.426407,84.852814);
\draw[red,ultra thick] plot[smooth,scale=14.142136] file {bumpingcurve.txt};
\end{scope}
\begin{scope} 
\clip[](35,0) -- (35,1) -- (30,1) -- (30,3) -- (27,3) -- (27,4) -- (25,4) -- (25,5) -- (23,5) -- (23,6) -- (21,6) -- (21,7) -- (19,7) -- (19,8) -- (17,8) -- (17,9) -- (15,9) -- (15,11) -- (14,11) -- (14,12) -- (13,12) -- (13,13) -- (12,13) -- (12,14) -- (11,14) -- (11,16) -- (9,16) -- (9,18) -- (7,18) -- (7,19) -- (6,19) -- (6,21) -- (5,21) -- (5,25) -- (4,25) -- (4,26) -- (3,26) -- (3,31) -- (2,31) -- (2,34) -- (0,34) -- (0,0);
\draw[black!30] (0,0) grid (35,34);
\end{scope}
\draw[](35,0) -- (35,1) -- (30,1) -- (30,3) -- (27,3) -- (27,4) -- (25,4) -- (25,5) -- (23,5) -- (23,6) -- (21,6) -- (21,7) -- (19,7) -- (19,8) -- (17,8) -- (17,9) -- (15,9) -- (15,11) -- (14,11) -- (14,12) -- (13,12) -- (13,13) -- (12,13) -- (12,14) -- (11,14) -- (11,16) -- (9,16) -- (9,18) -- (7,18) -- (7,19) -- (6,19) -- (6,21) -- (5,21) -- (5,25) -- (4,25) -- (4,26) -- (3,26) -- (3,31) -- (2,31) -- (2,34) -- (0,34) -- (0,0) -- cycle ;
\traj{27}

%\draw (15,-7) node {$A=1$};
\draw(20,17) node {$A=1$}; 
\end{tikzpicture}
\label{traj:A}
}
\hfill
\subfloat[]{
\begin{tikzpicture}[scale=0.08]
\begin{scope}
\clip (0,0) rectangle (42.426407,84.852814);
\draw[red,ultra thick] plot[smooth,scale=14.142136] file {bumpingcurve.txt};
\end{scope}
\begin{scope}
\clip[](53,0) -- (53,1) -- (51,1) -- (51,2) -- (44,2) -- (44,3) -- (41,3) -- (41,4) -- (36,4) -- (36,5) -- (35,5) -- (35,6) -- (33,6) -- (33,7) -- (31,7) -- (31,8) -- (30,8) -- (30,9) -- (28,9) -- (28,10) -- (26,10) -- (26,12) -- (24,12) -- (24,13) -- (22,13) -- (22,15) -- (19,15) -- (19,17) -- (18,17) -- (18,18) -- (17,18) -- (17,20) -- (15,20) -- (15,22) -- (14,22) -- (14,24) -- (13,24) -- (13,26) -- (11,26) -- (11,28) -- (10,28) -- (10,29) -- (9,29) -- (9,32) -- (7,32) -- (7,33) -- (6,33) -- (6,36) -- (5,36) -- (5,39) -- (4,39) -- (4,41) -- (2,41) -- (2,47) -- (1,47) -- (1,52) -- (0,52) -- (0,0);
\draw[black!30] (0,0) grid (53,52);
\end{scope}
\draw[](53,0) -- (53,1) -- (51,1) -- (51,2) -- (44,2) -- (44,3) -- (41,3) -- (41,4) -- (36,4) -- (36,5) -- (35,5) -- (35,6) -- (33,6) -- (33,7) -- (31,7) -- (31,8) -- (30,8) -- (30,9) -- (28,9) -- (28,10) -- (26,10) -- (26,12) -- (24,12) -- (24,13) -- (22,13) -- (22,15) -- (19,15) -- (19,17) -- (18,17) -- (18,18) -- (17,18) -- (17,20) -- (15,20) -- (15,22) -- (14,22) -- (14,24) -- (13,24) -- (13,26) -- (11,26) -- (11,28) -- (10,28) -- (10,29) -- (9,29) -- (9,32) -- (7,32) -- (7,33) -- (6,33) -- (6,36) -- (5,36) -- (5,39) -- (4,39) -- (4,41) -- (2,41) -- (2,47) -- (1,47) -- (1,52) -- (0,52) -- (0,0) -- cycle ;
\traj{27,25,21,21,16,16,15,15,13,13,12,12,11,11}

%\draw (25,-7) node {$A=2$}; 
\draw (30,20) node {$A=2$};
\end{tikzpicture}
\label{traj:B}
}
\hfill
\subfloat[]{
\begin{tikzpicture}[scale=0.08]
\begin{scope}
\clip (0,0) rectangle (42.426407,84.852814);
\draw[red,ultra thick] plot[smooth,scale=14.142136] file {bumpingcurve.txt};
\end{scope}
\begin{scope}
\clip[](80,0) -- (80,1) -- (73,1) -- (73,2) -- (65,2) -- (65,3) -- (63,3) -- (63,4) -- (61,4) -- (61,5) -- (58,5) -- (58,6) -- (53,6) -- (53,7) -- (50,7) -- (50,8) -- (49,8) -- (49,9) -- (47,9) -- (47,10) -- (44,10) -- (44,12) -- (42,12) -- (42,13) -- (39,13) -- (39,15) -- (38,15) -- (38,17) -- (36,17) -- (36,18) -- (33,18) -- (33,20) -- (29,20) -- (29,23) -- (28,23) -- (28,24) -- (27,24) -- (27,25) -- (25,25) -- (25,27) -- (23,27) -- (23,28) -- (22,28) -- (22,29) -- (21,29) -- (21,30) -- (20,30) -- (20,33) -- (18,33) -- (18,34) -- (17,34) -- (17,36) -- (16,36) -- (16,37) -- (15,37) -- (15,39) -- (14,39) -- (14,40) -- (13,40) -- (13,42) -- (12,42) -- (12,44) -- (9,44) -- (9,47) -- (8,47) -- (8,49) -- (7,49) -- (7,52) -- (6,52) -- (6,54) -- (5,54) -- (5,56) -- (4,56) -- (4,62) -- (3,62) -- (3,67) -- (2,67) -- (2,70) -- (1,70) -- (1,74) -- (0,74) -- (0,0);
\draw[black!30] (0,0) grid (80,74);
\end{scope}
\draw[](80,0) -- (80,1) -- (73,1) -- (73,2) -- (65,2) -- (65,3) -- (63,3) -- (63,4) -- (61,4) -- (61,5) -- (58,5) -- (58,6) -- (53,6) -- (53,7) -- (50,7) -- (50,8) -- (49,8) -- (49,9) -- (47,9) -- (47,10) -- (44,10) -- (44,12) -- (42,12) -- (42,13) -- (39,13) -- (39,15) -- (38,15) -- (38,17) -- (36,17) -- (36,18) -- (33,18) -- (33,20) -- (29,20) -- (29,23) -- (28,23) -- (28,24) -- (27,24) -- (27,25) -- (25,25) -- (25,27) -- (23,27) -- (23,28) -- (22,28) -- (22,29) -- (21,29) -- (21,30) -- (20,30) -- (20,33) -- (18,33) -- (18,34) -- (17,34) -- (17,36) -- (16,36) -- (16,37) -- (15,37) -- (15,39) -- (14,39) -- (14,40) -- (13,40) -- (13,42) -- (12,42) -- (12,44) -- (9,44) -- (9,47) -- (8,47) -- (8,49) -- (7,49) -- (7,52) -- (6,52) -- (6,54) -- (5,54) -- (5,56) -- (4,56) -- (4,62) -- (3,62) -- (3,67) -- (2,67) -- (2,70) -- (1,70) -- (1,74) -- (0,74) -- (0,0) -- cycle ;
\traj{27,25,21,21,16,16,15,15,13,13,12,12,11,11,11,11,9,9,9,9,9,9,9,9,8,8,8,8,8}

%\draw (35,-7) node {$A=3$};
\draw (45,20) node {$A=3$}; 
\end{tikzpicture}
\label{traj:C}
}

}


\newcommand{\traj}[1]
{
	\foreach \y [count=\j] in {#1} 
	{
		\draw[fill=CiemnoZielony!30] (\y,\j-1) rectangle +(1,1);
	}
}
\newtheorem{thm}{Theorem}
\newtheorem{lem}{Lemma}
\newtheorem{theorem}{Theorem}
\DeclareMathOperator{\Pos}{Pos}
\DeclareMathOperator{\boxi}{box}
\newcommand{\floor}[1]{\lfloor #1 \rfloor}
\newcommand{\strzalka}{\hspace{-3pt}\uparrow}

\newcommand{\rysuj}
{	\begin{center}
		\subfloat[]{
			\begin{tikzpicture}[scale=0.9]
				\diagram{4,3,2}
			\end{tikzpicture}
			\label{fig:rysa}
		}
		\hspace{30pt}
		\subfloat[]{
			\begin{tikzpicture}[scale=0.9]
				\tableau{{1,2,4,7},{3,6,9},{5,8}}
			\end{tikzpicture}
		 \label{fig:rysb}	
		 }
	\end{center}
}

\newcommand{\steprsk}
{
\centering
\hfill
\subfloat[]{
	\begin{tikzpicture}[scale=0.75]
	%\draw[ultra thick] (10,0) -- (0,0) -- (0,10) ;
	%\fill[red,fill=red,opacity=0.4] (0,0) rectangle +(10,1);
	\clip (-1.5,-0.5) rectangle (4.5,4.5);
	%	\fill[blue!10] (1,0) rectangle +(1,1);
	%	\fill[blue!10] (1,1) rectangle +(1,1);
	%	\fill[blue!10] (0,2) rectangle +(1,1);
	
	
	\draw (0,0) rectangle +(1,1); 
	\node at (0.5,0.5) {16}; 
	\draw (1,0) rectangle +(1,1); 
	\node at (1.5,0.5) {37}; 
	\draw (2,0) rectangle +(1,1); 
	\node at (2.5,0.5) {41}; 
	\draw (3,0) rectangle +(1,1); 
	\node at (3.5,0.5) {82}; 
	\draw (0,1) rectangle +(1,1); 
	\node at (0.5,1.5) {23}; 
	\draw (1,1) rectangle +(1,1); 
	\node at (1.5,1.5) {53}; 
	\draw (2,1) rectangle +(1,1); 
	\node at (2.5,1.5) {70}; 
	\draw (0,2) rectangle +(1,1); 
	\node at (0.5,2.5) {74}; 
	\draw (1,2) rectangle +(1,1); 
	\node at (1.5,2.5) {99}; 
	
	
	%	\draw[->] (-0.3,-0.45) to[bend left=60] (-0.3,0.45);
	%	\draw[->] (0,1) +(-0.3,-0.45) to[bend left=60] +(-0.3,0.45);
	%	\draw[->] (0,2) +(-0.3,-0.45) to[bend left=60] +(-0.3,0.45);
	%	\draw[->] (0,3) +(-0.3,-0.45) to[bend left=60] +(-0.3,0.45);
	%	
	%	\tiny
	%	\node[] at (-1,0) {18};
	%	\node[] at (-1,1) {37};
	%	\node[] at (-1,2) {53};
	%	\node[] at (-1,3) {74};
	%	
	\end{tikzpicture}
	\label{fig:RSKa}
}
\hfill
\subfloat[]{
\begin{tikzpicture}[scale=0.75]
%\draw[ultra thick] (10,0) -- (0,0) -- (0,10) ;
%\fill[red,fill=red,opacity=0.4] (0,0) rectangle +(10,1);
\clip (-1.5,-0.5) rectangle (4.5,4.5);
\fill[blue!10] (1,0) rectangle +(1,1);
\fill[blue!10] (1,1) rectangle +(1,1);
\fill[blue!10] (0,2) rectangle +(1,1);


\draw (0,0) rectangle +(1,1); 
\node at (0.5,0.5) {16}; 
\draw (1,0) rectangle +(1,1); 
\node at (1.5,0.5) {37}; 
\draw (2,0) rectangle +(1,1); 
\node at (2.5,0.5) {41}; 
\draw (3,0) rectangle +(1,1); 
\node at (3.5,0.5) {82}; 
\draw (0,1) rectangle +(1,1); 
\node at (0.5,1.5) {23}; 
\draw (1,1) rectangle +(1,1); 
\node at (1.5,1.5) {53}; 
\draw (2,1) rectangle +(1,1); 
\node at (2.5,1.5) {70}; 
\draw (0,2) rectangle +(1,1); 
\node at (0.5,2.5) {74}; 
\draw (1,2) rectangle +(1,1); 
\node at (1.5,2.5) {99}; 


\draw[->] (-0.3,-0.45) to[bend left=60] (-0.3,0.45);
\draw[->] (0,1) +(-0.3,-0.45) to[bend left=60] +(-0.3,0.45);
\draw[->] (0,2) +(-0.3,-0.45) to[bend left=60] +(-0.3,0.45);
\draw[->] (0,3) +(-0.3,-0.45) to[bend left=60] +(-0.3,0.45);

\tiny
\node[] at (-1,0) {18};
\node[] at (-1,1) {37};
\node[] at (-1,2) {53};
\node[] at (-1,3) {74};

\end{tikzpicture}
\label{fig:RSKb}
}
\hfill
\subfloat[]{
	\begin{tikzpicture}[scale=0.75]
	%\draw[ultra thick] (10,0) -- (0,0) -- (0,10) ;
	%\fill[red,fill=red,opacity=0.4] (0,0) rectangle +(10,1);
	\clip (-1.5,-0.5) rectangle (4.5,4.5);
	\fill[blue!10] (1,0) rectangle +(1,1);
	\fill[blue!10] (1,1) rectangle +(1,1);
	\fill[blue!10] (0,2) rectangle +(1,1);
	\fill[blue!10] (0,3) rectangle +(1,1);
	
	\draw (0,0) rectangle +(1,1); 
	\node at (0.5,0.5) {16}; 
	\draw (1,0) rectangle +(1,1); 
	\node at (1.5,0.5) {18}; 
	\draw (2,0) rectangle +(1,1); 
	\node at (2.5,0.5) {41}; 
	\draw (3,0) rectangle +(1,1); 
	\node at (3.5,0.5) {82}; 
	\draw (0,1) rectangle +(1,1); 
	\node at (0.5,1.5) {23}; 
	\draw (1,1) rectangle +(1,1); 
	\node at (1.5,1.5) {37}; 
	\draw (2,1) rectangle +(1,1); 
	\node at (2.5,1.5) {70}; 
	\draw (0,2) rectangle +(1,1); 
	\node at (0.5,2.5) {53}; 
	\draw (1,2) rectangle +(1,1); 
	\node at (1.5,2.5) {99}; 
	\draw (0,3) rectangle +(1,1); 
	\node at (0.5,3.5) {74}; 
	
	
	
	\draw[->] (-0.3,-0.45) to[bend left=60] (-0.3,0.45);
	\draw[->] (0,1) +(-0.3,-0.45) to[bend left=60] +(-0.3,0.45);
	\draw[->] (0,2) +(-0.3,-0.45) to[bend left=60] +(-0.3,0.45);
	\draw[->] (0,3) +(-0.3,-0.45) to[bend left=60] +(-0.3,0.45);
	
	\tiny
	\node[] at (-1,0) {18};
	\node[] at (-1,1) {37};
	\node[] at (-1,2) {53};
	\node[] at (-1,3) {74};
	
	\end{tikzpicture}
	\label{fig:RSKc}
}


}

\newcommand{\tableau}[1]
{
	\foreach \x [count=\i] in {#1} 
	{
		\foreach \y [count=\j] in \x 
		{
			\draw[ultra thick] (\j+1,0) -- (0,0) -- (0,\i+0.5);
			\draw (\j-1,\i-1) rectangle +(1,1); 
			\node[] at (\j-0.5,\i-0.5) {\y};
		}	
	}
}

\newcommand{\diagram}[1]
{
	\foreach \x [count=\i] in {#1} 
	{
		\draw[ultra thick] (\x+1,0) -- (0,0) -- (0,\i+0.5);
		\foreach \j in {1,...,\x}
		\draw (\j-1,\i-1) rectangle +(1,1); 
	}
}

%% define your title in the usual way
\title[Hydrodynamic limit of RSK]{Hydrodynamic limit \\ of the Robinson--Schensted--Knuth algorithm}

%% define your authors in the usual way
%% use \addressmark{1}, \addressmark{2} etc for the institutions, and use \thanks{} for contact details
\author[Mikołaj Marciniak]{Mikołaj Marciniak\thanks{\href{mailto:marciniak@mat.umk.pl}{marciniak@mat.umk.pl}. Mikołaj Marciniak was partially supported by Narodowe Centrum Nauki, grant number 2017/26/A/ST1/00189.}\addressmark{1}}
%% cudzyslow “...		”
%% then use \addressmark to match authors to institutions here
\address{\addressmark{1}Interdisciplinary Doctoral School “Academia Copernicana”, Faculty of Mathematics and Computer Science, Nicolaus Copernicus University in Toruń, ul.~Chopina 12/18, 87-100 Toruń, Poland}

%% put the date of submission here
\received{\today}

%% leave this blank until submitting a revised version
%\revised{}

%% put your English abstract here, or comment this out if you don't have one yet
%% please don't use custom commands in your abstract / resume, as these will be displayed online
%% likewise for citations -- please don't use \cite, and instead write out your citation as something like (author year)
\abstract{We investigate the evolution in time of the position of a fixed number in the insertion tableau when the Robinson--Schensted--Knuth algorithm is applied to a sequence of random numbers. When the length of the sequence tends to infinity, a~typical trajectory after scaling converges uniformly in probability to some deterministic curve.}

%% put your French abstract here, or comment this out if you don't have one
%%\resume{\lipsum[2]}

%% put your keywords here, or comment this out if you don't have them yet
\keywords{RSK algorithm, bumping route, random Young tableaux, limit shape}

%% you can include your bibliography however you want, but using an external .bib file is STRONGLY RECOMMENDED and will make the editor's life much easier
%% regardless of how you do it, please use numerical citations, ie. [xx, yy] in the text

%% this sample uses biblatex, which (among other things) takes care of URLs in a more flexible way than bibtex
%% but you can use bibtex if you want
%\usepackage[backend=bibtex]{biblatex}
%\addbibresource{route.bib}
%% note the \printbibliography command at the end of the file which goes with these biblatex commands


\begin{document}

%\todo[inline]{właściwym sposobem regulowania odstępów między paragrafami są komendy \texttt{smallskip}, \texttt{medskip}, \texttt{bigskip}}
\todo[inline]{nazwac krzywe: "Position routes"?, być może warto nie nazywać granicy hydrodynamiczną}
	
	
\maketitle
%% note that you DO NOT have to put your abstract here -- it is generated by \maketitle and the \abstract and \resume commands above

\section{Introduction}

\subsection{Notations}
A \emph{partition} of a natural number $n$ is a break up of $n$ into a sum $n=\lambda_1+\lambda_2+\cdots+\lambda_k$ where $\lambda_1\geq \lambda_2\geq\cdots\geq\lambda_k>0$ are positive integer numbers. The vector \linebreak[4]{$\lambda=(\lambda_1,\lambda_2,\ldots,\lambda_k)$} is usually used to denote a partition. Let $\lambda\vdash n$ denote that $\lambda$ is a partition of a number $n$. A \emph{Young diagram} $\lambda=(\lambda_1,\lambda_2,\ldots,\lambda_k)$ is a finite collection of boxes arranged in left-justified rows with the row length $\lambda_j$ of the $j$-th row. Thus the Young diagram $\lambda$ is a graphical interpretation of the partition $\lambda$. A \emph{Young tableau} is a Young diagram filled with numbers. If the entries strictly decrease along each column from top to bottom and weakly increase along each row from left to right, a tableau is called \emph{semistandard}. A \emph{standard Young tableau} is a semistandard Young tableau with $n$ boxes which contains all numbers $1, 2, \ldots, n$. \cref{fig:diagtab} shows examples of a Young diagram and of a standard Young tableau.

\begin{figure}[h]
\rysuj
\caption{\protect\subref{fig:rysa} The Young diagram of shape $(4,3,2)\vdash 9$ and \protect\subref{fig:rysb} a standard Young tableau of shape $(4,3,2)\vdash 9$. }
\label{fig:diagtab}
\end{figure}



The \emph{Robinson--Schensted--Knuth algorithm RSK} is a bijective algorithm which takes a finite sequence of numbers of length $n$ as the input and returns a pair of Young tableaux $(P,Q)$  with the same shape $\lambda\vdash n$. The semistandard tableau $P$ is called an \emph{insertion tableau}, and the standard tableau $Q$ is called a \emph{recording tableau}. In particular, the RSK algorithm assigns to any permutation $\sigma$ a pair $(P,Q)$ of standard Young tableaux. A detailed description of the RSK algorithm can be found in \cite[Chapter 1.6]{Romik2015}. 


The RSK algorithm is based on applying the \emph{insertion step} to successive numbers from a given finite sequence $\{X_j\}_{j=1}^n$. The insertion step takes as input the previously obtained tableau $P(X_1, X_2, \ldots, X_{j-1})$ and the next number $X_j$ from the sequence. It produces as the output a new tableau $P(X_1, X_2, \ldots, X_j)$ with shape ``increased'' by one box; this tableau is obtained in the following way, see \cref{fig:RSK}. The RSK-insertion step starts in the first row with the number $x:=X_j$. The insertion step consists of inserting the number $x$ into the leftmost box in this row containing a number $y$ greater than $x$. Move to the next row with the number $x:=y$ and repeat the action. At some row we are forced to insert the number at the end of the row, which will end the insertion step. The collection of rearranged boxes is called the \emph{bumping route}. 

\begin{figure}[h]
\steprsk
\caption{\protect\subref{fig:RSKa} The original tableau $P$.
	\protect\subref{fig:RSKb} The highligted boxes form the bumping route which
	corresponds to an insertion of the number $18$. The numbers next to the arrows indicate the bumped numbers.\protect\subref{fig:RSKc} The
	output of the RSK insertion step. }
\label{fig:RSK}
\end{figure}

We can say that the boxes with numbers are moved along the bumping route during an RSK insertion step. 
%The insertion step of the RSK algorithm can change the position of some numbers. 
In this article we will investigate the position of the box with a selected number and how this position is changing over time. 
\todo[inline]{nadać nazwę i zdefioniować}


\subsection{Motivations}
The RSK algorithm is an important tool in algebraic combinatorics, especially in the context of Littlewood--Richardson coefficients and the plactic monoid\cite[Lecture 4]{Fulton1991}. 

For many years mathematicians have been studying the asymptotic behavior of the insertion tableau when we apply the Robinson--Schensted--Knuth algorithm to a random input. In the following paragraphs we will see several examples of such considerations. 

The Ulam--Hammersley problem \cite[Chapter 1.1]{Romik2015} concerns the typical length of the longest increasing subsequence in a random permutation. This problem corresponds to the problem of finding the typical length of the first row in the Young tableau obtained by the RSK algorithm from the sequence of independent random variables $\{X_j\}_{j=1}^n$ with the uniform distribution $U(0,1)$ on the unit interval $(0,1)$.

More generally, the RSK algorithm applied to the sequence of independent and identically distributed random variables with the uniform distribution $U(0,1)$ on the unit interval $(0,1)$ generates the \emph{Plancherel measure} on Young diagrams \cite[Chapter 1.8]{Romik2015}. The Plancherel measure is an important element of the representation theory of the symmetric groups because it describes how the left regular representation decomposes into irreducible components \cite[Chapter 3.3]{Fulton1991}.

Logan and Shepp \cite{LoganShepp1977} and Vershik and Kerov \cite{VershikKerov1986} described the limit shape of the insertion tableau $P(X_1, X_2, \ldots, X_n)$ obtained when we apply the RSK algorithm to a random finite sequence.

Romik and Śniady \cite{Romik2016} considered the limit shape of the bumping routes obtained when applying an RSK insertion step with a fixed number $w$ to an existing insertion tableau $P(X_1, X_2, \ldots, X_n)$ obtained from a random finite sequence. In \cite{Romik2015a} they also considered the limit shape of \emph{jeu de taquin} obtained from the recording tableau $Q(w, X_1, X_2, \ldots, X_n)$ made from a random finite sequence preceded by a fixed number $w$. 

\subsection{The main problem}
This paper also concerns the asymptotic behavior of the insertion tableau when we apply the RSK algorithm to a random input. \emph{What can we say about the evolution over time of the insertion tableau from the viewpoint of box dynamics, when we apply the RSK algorithm to a sequence of independent random variables with the uniform distribution $U(0,1)$? How do the boxes move in the insertion tableau? If we investigate the scaled position of a box with a fixed number, will we get a deterministic limit, when the number of boxes tends to infinity?}

More specifically, we consider the insertion tableau $P(X_1, X_2, \ldots, X_n, w, X_{n+1}, \ldots, X_m)$ obtained by the RSK algorithm applied to a random finite sequence containing a fixed number $w$ at some index. The box with this fixed number $w$ is being bumped by the RSK insertion step along the bumping routes. We will describe the scaled limit position of the box with the number $w$ depending on the ratio of the numbers $m$ and $n$. This problem was also stated by Duzhin \cite{Duzhin2019}.
\section{Tools}
\subsection{The function $G(x)$}
\label{wzory}
Just like Romik and Śniady in \cite[Chapter 5.1]{Romik2015a} we define the functions $F_{SC}, \Omega_{\star}, u, v$ needed to describe the macroscopic position of the new box in the RSK insertion step and additionally the function $G$
\begin{align*}
F_{SC}(y)&=\frac{1}{2}+\frac{1}{\pi}\bigg(\frac{y\sqrt{4-y^2}}{4}+\sin^{-1}\Big(\frac{y}{2}\Big)\bigg)\hspace{30pt}&\big(-2\leq y \leq 2\big),\\
\Omega_{\star}(y)&=\frac{2}{\pi}\bigg(\sqrt{4-y^2}+y\sin^{-1}\Big(\frac{y}{2}\Big)\bigg)&\big(-2\leq y \leq 2\big),\\
u(x)&=F_{SC}^{-1}(x) \hspace{30pt}&\big(0\leq x \leq 1\big),\\
v(x)&=\Omega_{\star}(u(x)) \hspace{30pt}&\big(0\leq x \leq 1\big),\\
G(x)&=\Big(\frac{v(x)+u(x)}{2},\frac{v(x)-u(x)}{2}\Big) \hspace{30pt}&\big(0\leq x \leq 1\big).
\end{align*}
The function $F_{SC}$ is the cumulative distribution function of the \emph{Wigner's semicircle distribution}. The function $\Omega_{\star}$ is the limit shape of the Young tableau sampled from the \emph{Plancherel measure}. This curve is called the Logan--Shepp--Vershik--Kerov curve \cite{LoganShepp1977, VershikKerov1986}. The function $x \mapsto \big(u(x), v(x)\big)$ is a special parameterisation of the function $\Omega_{\star}$ and describes the limit position of the new box in the RSK insertion step applied with the number $x$ to the random Young tableau \cite[Chapter 5.1]{Romik2015a}. The function $G\colon [0,1] \to \mathbb{R}_+^2$ is the function $\big(u(x), v(x)\big)$ rotated by 45 degrees (see \cite[Chapter 5.1]{Romik2015a}). The function $G$ is continuous.
\begin{figure}
 \begin{center}
\verker
\end{center}
\caption{The graph of the function $G(x)$ with specified values for the argument\\
$x=0.0$, $x=0.2$, $x=0.4$, $x=0.6$ $x=0.8$ and $x=1.0$.}
\label{fig:gfunction}
\end{figure}
\cref{fig:gfunction} shows the graph of the function $G\colon [0,1] \to \mathbb{R}_+^2$. \\
\pagebreak[4] 
\subsection{The result of Romik and Śniady}
In the proof of \cref{thm:mainthm} we will need the following result of Romik and Śniady \cite[Theorem 5.1]{Romik2015a}. 

Let $\{X_j\}_{j=1}^\infty$ be a  sequence of independent random variables with the uniform distribution $U(0,1)$. Let $\square_n(x)\in \mathbb{N}^2$ denote the position of the box with the maximal number in the recording tableau $Q(X_1, \ldots, X_n, x)$ when we apply the RSK insertion step for the number $x\in[0,1]$ to the previously obtained tableau $P(X_1, \ldots, X_n)$. 
\begin{theorem}
\label{thm:romthm}
For each $x\in[0,1]$ the position $\square_n(x)$, after scaling by $\frac{1}{\sqrt{n}}$, converges in probability to a specific point $G(x)\in \mathbb{R}_+^2$, when $n$ tends to infinity: $$\frac{\square_n(x)}{\sqrt{n}}\overset{p}{\to}G(x).$$ 
\end{theorem}

\subsection{The partial order on the plane}
\label{order}
We define \emph{the partial order $\prec$ on the plane} as follows:
$(x_1,y_1)\prec(x_2,y_2)$ if and only if $x_1\leq x_2$ and $y_1\geq y_2$. 

For example the function $G$ is \emph{increasing with respect to the relation $\prec$}, i.e. if $x_1\leq x_2$ then $G(x_1)\prec G(x_2)$. Likewise, from a property of the Insertion tableau (each row of is non-decreasing), for each natural number $n\in \mathbb{N}$ the function $\square_n$ is increasing with respect to the relation $\prec$, i.e. if $x_1\leq x_2$ then $\square_n(x_1)\prec \square_n(x_2)$.
%\todo[inline]{wyjaśnij z jakiej właśności konkretnie korzystasz}
\subsection{The random increasing sequence}
\label{incseq}
Let $w\in(0,1]$ be a fixed number and let $\{X'_j\}_{j=1}^{m'}$ be a finite sequence of independent random variables with the uniform distribution $U(0,w)$. For $j\in\{1, 2, \ldots, m'\}$ we define $z(j)$ as the $j$-th order statistic, i.e. the $j$-th smallest number among $X'_1, X'_2, \ldots, X'_{m'}$. The sequence $\{z(j)\}_{j=1}^{m'}$ contains all elements of the sequence $\{X'_j\}_{j=1}^{m'}$ in the ascending order. The sequence $\{z(j)\}_{j=1}^{m'}$ will be called a \emph{random increasing sequence with the uniform distribution on the interval $[0,w]$}. 

In addition, the sequence $\{X'_j\}_{j=1}^{m'}$ is some permutation $\Pi=\left(\Pi_1, \Pi_2, \ldots, \Pi_{m'}\right)$ of the sequence $\{z(j)\}_{j=1}^{m'}$. Thus
$$\Big\{X'_j\Big\}_{j=1}^{m'}=\Big\{z(\Pi_j)\Big\}_{j=1}^{m'}=z\circ\Pi,$$
where $z$ is the function that acts pointwise on every element of the permutation $\Pi$. The permutation $\Pi$ is a random permutation with the uniform distribution. We see that a finite sequence of independent random variables with the uniform distribution $U(0,w)$ has the same distribution as a finite random increasing sequence with the uniform distribution on the interval $[0,w]$ permuted by a random permutation with the uniform distribution.

\section{The main result}

\subsection{Statement of Main Result}
Our main result is \cref{thm:mainthm} describing the asymptotic behavior of the box with a fixed number. It states that when the number of boxes tends to infinity then the (scaled down) trajectory of the box converges in probability to the curve
$H\colon [1,\infty) \to \mathbb{R}_+^2$ given by
$$H(T):=\sqrt{T}\ G\left(\frac{1}{T}\right).$$ 
The same curve $H$ also happens to be the limit shape of the bumping routes \cite{Romik2016} in the RSK algorithm. \cref{fig:trzy} shows the graph of the curve $H$ and the experimentally determined trajectory of the box with the number $w=0.5$.


\begin{figure}[h]
\trzy
\caption{ \protect\subref{traj:A}
The initial shape of the insertion tableau $P(X_1,\dots,X_n,w)$ immediately after the new box
with the number $w$ was added (the higlighted box in the bottom row) for $n=400$ and $w=0.5$.
\protect\subref{traj:B}
The shape of the insertion tableau 
$P(X_1,\dots,X_n,w,X_{n+1},\dots,X_{\lfloor T n \rfloor})$
at the time parameter $T=2$. The highlighted boxes indicate the trajectory of the box with the number $w$. The red smooth curve is the plot of $H$.
\protect\subref{traj:C} Analogous picture for $T=3$.
}
\label{fig:trzy}
\end{figure}

\pagebreak[3]

More specifically, let $w\in(0,1]$ be a fixed number. Let $\{X_j\}_{j=1}^\infty$ be a  sequence of independent random variables with the uniform distribution $U(0,1)$ on the unit interval $[0,1]$. For every $n\in\mathbb{N}$ we define the function $\Pos_n:\{n+1, n+2, \ldots\}\to\mathbb{N}^2$ by:
$$\Pos_n\left(j\right)=\boxi_w\biggl(P\left(X_1, \ldots, X_n, w, X_{n+1}, \ldots, X_j\right)\biggr)$$ for $j\in\{n+1, n+2, \ldots\}$, where for a tableau $P$ we denote by  $\boxi_w(P)\in\mathbb{N}^2$ the coordinates of the box with the number $w$. 


\begin{theorem}
\label{thm:mainthm}
Let $R\in(1, \infty)$ be a real number. For each number $T\in[1,R]$ the random variable $\Pos_n(\floor{Tn})$, after scaling by $\frac{1}{\sqrt{wn}}$, converges in probability to $H(T)$, when $n$ tends to infinity. Moreover, the convergence is uniform, i.e.  for each $\epsilon>0$
$$	\lim_{n\to\infty}\mathbb{P}\Bigg(\sup_{T\in[1,R]} \left\|\frac{\Pos_n\left(\floor{Tn}\right)}{\sqrt{wn}}-H\left(T\right)\right\|>\epsilon\Bigg)=0.$$
\end{theorem}

%\todo[inline]{brakujący podrozdział tu lub gdzieś indziej}
\subsection{The strategy of the proof of \cref{thm:mainthm}}
The proof will be presented in consecutive subsections of this section. 
Here we present the sketch of the proof.

First, we prove the pointwise convergence in the following way.
\begin{itemize}
\item We skip the numbers that are larger than $w$.
\item We restrict our attention only to the permutation generated by the numbers.  
\item We use the property of the RSK algorithm that the insertion tableau is equal to the recording tableau of the inverse permutation.
\item We return to the sequence of the numbers from the interval $[0,1]$.
\item We apply the Romik-Sniady result about bumping routes to deduce the pointwise convergence.
 \end{itemize}
 
Second, using the second Dini theorem, we prove the uniform convergence.

\subsection{The pointwise convergence}

First, we prove only the pointwise convergence, i.e. we prove that for each number $T\in[1,\infty)$ and for each $\epsilon>0$
$$	\lim_{n\to\infty}\mathbb{P}\Bigg( \left\|\frac{\Pos_n\left(\floor{Tn}\right)}{\sqrt{wn}}-H\left(T\right)\right\|>\epsilon\Bigg)=0.$$ 

\begin{proof}
We apply the RSK algorithm to a random sequence of real numbers containing the number $w$ and investigate the position of the box with the number $w$ in the insertion tableau. Any insertion step applied to a number greater that $w$ does not change the position of the number $w$ in the tableau, so it is enough to consider only the subsequence containing numbers not greater than $w$. 

Now we will use this observation in the proof. Let $m=\floor{Tn}$. The probability that the same number occurs twice in the sequence $(w, X_1, X_2, \ldots)$ is equal to 0, hence without losing generality we assume that the numbers $w, X_1, X_2, \ldots$ are all different. Let $\{X'_j\}_{j=1}^\infty$ be the subsequence of the sequence $\{X_j\}_{j=1}^\infty$ containing all elements of the sequence $\{X_j\}_{j=1}^\infty$ which are less than $w$. The sequence $\{X'_j\}_{j=1}^{\infty}$ is a sequence of independent random variables with the uniform distribution $U(0,w)$. 

Let $n'=n'(n)$ and $m'=m'(n)$ denote the number of elements, respectively, of the sequences $\{X_j\}_{j=1}^n$ and $\{X_j\}_{j=1}^m$ which are smaller than $w$. Then there is an equality:
\begin{align*}
\Pos_n\Big(\floor{Tn}\Big)&=\boxi_w\Big(P\left(X_1, \ldots, X_n, w, X_{n+1}, \ldots, X_{m}\right)\Big)\\&=\boxi_w\Big(P\left(X'_1, \ldots, X'_{n'}, w, X'_{n'+1}, \ldots, X'_{m'}\right)\Big).
\end{align*}
The random variable $n'=\sum_{j=1}^{n}[X_j<w]$ counts how many numbers from the sequence $\{X_j\}_{j=1}^{n}$ are smaller than $w$, so $n'$ is a random variable with the binomial distribution with parameters $n$ and $w$. We denote it $n'\sim B(n,w)$. Likewise, the random variable $m'-n'$ counts how many numbers from the sequence $\{X_j\}_{j=n+1}^{m}$ are smaller than $w$, so \linebreak[4]{$m'-n'\sim B(m-n,w)$.} Moreover, the random variables $n'$ and $m'-n'$ are independent, because the random variables $X_1, X_2, \ldots$ are independent. 

From the Strong Law of Large Numbers \cite[Theorem 2.4.1]{Durrett2019} we know that if $n$ tends to infinity then the following limits exist almost surely:
\begin{align*}
\lim_{n\to\infty}\frac{n'}{n}&=\mathbb{E}[X_j<w]=\mathbb{P}(X_j<w)= w, \\
\lim_{n\to\infty}\frac{m'-n'}{m-n}&=\mathbb{E}[X_j<w]=\mathbb{P}(X_j<w)=w. 
\end{align*}
Therefore, also the following limits exist almost surely
\begin{align*}
\lim_{n\to\infty}\frac{m'}{n'}
&= 1+\lim_{n\to\infty}\frac{m'-n'}{n'}\\
&= 1+\lim_{n\to\infty}\frac{m-n}{n}\\
%&=1+\lim_{n\to\infty}\frac{m'-n'}{m-n}\frac{1}{\frac{n'}{n}}\frac{m-n}{n}\\
%&=1+\lim_{n\to\infty}\frac{m'-n'}{m-n}\frac{1}{\frac{n'}{n}}\left(\frac{m}{n}-1\right)\\
%&=1+\frac{w}{w}\left(T-1\right)\\
&=T
\end{align*}
and
\begin{align*}
\lim_{n\to\infty}\frac{m'}{n}&=\lim_{n\to\infty}\frac{n'}{n}\frac{m'}{n'}\\&=wT.
\end{align*}

We define the function $z:\{1, 2, \ldots m'\}\cup\{m'+1, n'+\frac{1}{2}\}\to [0,1]$ that assigns to a number $j\in\{1, 2, \ldots, m'\}$, the $j$-th smallest number among $X'_1, X'_2, \ldots, X'_{m'}$ and additionally $z\left(m'+1\right)=w$ and $z\left(n'+\frac{1}{2}\right)=\frac{z\left(n'\right)+z\left(n'+1\right)}{2}$. 

From the property of the random increasing sequence with the uniform distribution (\cref{incseq}) we have:
$$\Big\{X'_j\Big\}_{j=1}^{m'}=\Big\{z(\Pi_j)\Big\}_{j=1}^{m'}=z\circ\Pi,$$
where $\{z(j)\}_{j=1}^{m'}$ is a random increasing sequence with the uniform distribution on the interval $[0,w]$ and $\Pi$ is a random permutation of range $m'$ with the uniform distribution. The function $z$ acts pointwise on every element of the permutation $\Pi$. 
Similarly, the function $z$ acts on the insertion tableau by acting on each box individually. The function $z$ by acting on the input of the RSK algorithm change pointwise the numbers in the insertion tableau, since the function $z$ is increasing. Then
\begin{align*}
\Pos_n\left(m\right)&=\boxi_w\Big(P\left(X'_1, \ldots, X'_{n'}, w, X'_{n'+1}, \ldots, X'_{m'}\right)\Big)
\\&= \boxi_w\bigg(P\Big(z(\Pi_1), \ldots, z\left(\Pi_{n'}\right), w, z\left(\Pi_{n'+1}\right), \ldots, z\left(\Pi_{m'}\right)\Big)\bigg)
\\&=\boxi_w\Big(z\circ P\left(\Pi_1, \ldots, \Pi_{n'}, m'+1, \Pi_{n'+1}, \ldots, \Pi_{m'}\right)\Big)
\\&=\boxi_{m'+1}\Big(P\left(\Pi_1, \ldots, \Pi_{n'}, m'+1, \Pi_{n'+1}, \ldots, \Pi_{m'}\right)\Big).
\end{align*}
We denote the inverse permutation to $\Pi$ by $\Pi^{-1}=\Big(\Pi^{-1}_1, \ldots, \Pi^{-1}_{m'}\Big)$ and we define the permutation $$\Pi\strzalka=\Big(\Pi_1, \ldots, \Pi_{n'}, m'+1, \Pi_{n'+1}, \ldots, \Pi_{m'}\Big)$$ as a natural extension of the permutation $\Pi$. Then $\Pi\strzalka^{-1}=\Big(\Pi\strzalka^{-1}_1, \ldots, \Pi\strzalka^{-1}_{m'+1}\Big)$ where
$$
\Pi\strzalka^{-1}_j=
\begin{cases}
\Pi_j^{-1}&\text{if}\hspace{10pt}\Pi_j^{-1}\leq n', \\
\Pi_j^{-1}+1& \text{if}\hspace{10pt}n'<\Pi_j^{-1}\leq m', \\
n'+1& \text{if}\hspace{10pt}j=m'+1. 
\end{cases}
$$ 
In addition, we will use the fact \cite{Schuetzenberger1963} that for any permutation $\Pi\strzalka$ the insertion tableau of $\Pi\strzalka$ is equal to the recording tableau of the inverse permutation $\Pi\strzalka^{-1}$:
$$P\left(\Pi\strzalka\right)=Q\left(\Pi\strzalka^{-1}\right).$$
Therefore
\begin{align*}
\Pos_n\left(m\right)&=\boxi_{m'+1}\bigg(P\left(\Pi_1, \ldots, \Pi_{n'}, m'+1, \Pi_{n'+1}, \ldots, \Pi_{m'}\right)\bigg)\\&=\boxi_{m'+1}\bigg(Q\left(\Pi\strzalka^{-1}\right)\bigg)\\&=\boxi_{m'+1}\bigg(Q\left(\Pi\strzalka^{-1}_1, \ldots, \Pi\strzalka^{-1}_{m'}, n'+1\right)\bigg).
\end{align*}
Now, by replacing the number $n'+1$ with the number $n'+\frac{1}{2}$ we get the inverse permutation $\Pi^{-1}$ %\todo{na pewno with?}
\begin{align*}
\Pos_n\left(m\right)&=\boxi_{m'+1}\bigg(Q\Big(\Pi\strzalka^{-1}_1, \ldots, \Pi\strzalka^{-1}_{m'}, n'+1\Big)\bigg)\hspace{50pt}
\\&=\boxi_{m'+1}\bigg(Q\Big(\Pi\strzalka^{-1}_1, \ldots, \Pi\strzalka^{-1}_{m'}, n'+\frac{1}{2}\Big)\bigg)
\\&=\boxi_{m'}\bigg(Q\Big(\Pi^{-1}_1, \ldots, \Pi^{-1}_{m'}, n'+\frac{1}{2}\Big)\bigg).
\end{align*}
%\todo[inline]{oczyszczenie dalszych dowodów - skrót końcówki " Essentially it just expresses thatpointwise convergence + monotonicity yields uniform convergence.  Is there a moresuccinct way to show this?" }

Since the function $z$ is increasing, the function $z$ by acting on the input of the RSK algorithm not change the content of the recording tableau. Using the function $z$, we will get a sequence of independent random variables with the uniform distribution $U(0,1)$. Indeed
\pagebreak[2]
\begin{align*}
\Pos_n\left(m\right)&=\boxi_{m'}\bigg(Q\Big(\Pi^{-1}_1, \ldots, \Pi^{-1}_{m'}, n'+\frac{1}{2}\Big)\bigg)
\\&=\boxi_{m'}\bigg(z\circ Q\Big(\Pi^{-1}_1, \ldots, \Pi^{-1}_{m'}, n'+\frac{1}{2}\Big)\bigg)
\\&=\boxi_{m'}\Bigg(Q\bigg(z\Big(\Pi^{-1}_1\Big), \ldots, z\Big(\Pi^{-1}_{m'}\Big), z\Big(n'+\frac{1}{2}\Big)\bigg)\Bigg).
\end{align*}

The permutation $\Pi$ is a random permutation with the uniform distribution on $S_n$, so $\Pi^{-1}$ is also a random permutation with the uniform distribution on $S_n$. If we act with a random permutation on a random increasing sequence with the uniform distribution we will get a sequence of  independent random variables with the uniform distribution, thus the sequence $$z\circ\Pi^{-1}=\bigg(z\Big(\Pi^{-1}_1\Big), \ldots, z\Big(\Pi^{-1}_{m'}\Big)\bigg)$$ is a sequence of independent random variables with the uniform distribution $U(0,w)$. \\

We define the random variable $A_n$ and the sequence $\{Y_j\}_{j=1}^{m'}$: 
\begin{align*}
Y_j &=\frac{z\left(\Pi_j^{-1}\right)}{w} \qquad \text{for }j\in{1, 2, \ldots, m'},\\
A_n &=\frac{z\left(n'+\frac{1}{2}\right)}{w}=\frac{z(n')+z(n'+1)}{2w}.
\end{align*}
The sequence $\{Y_j\}_{j=1}^{m'}$ is a sequence of independent random variables with the uniform distribution $U(0,1)$, and the random variable $A_n$ converges almost surely to the limit $A=\frac{1}{T}$:
\begin{align*}
\lim_{n\to\infty}A_n&=\lim_{n\to\infty}\frac{z\left(n'\right)+z\left(n'+1\right)}{2w}\\&=\lim_{n\to\infty}\frac{z\left(n'\right)}{w}+\lim_{n\to\infty}\frac{z\left(n'+1\right)-z\left(n'\right)}{2w}\\&=\lim_{n\to\infty}\frac{n'}{m'}+0\\&=\frac{1}{T}.
\end{align*}
Therefore
\begin{align*}
\Pos_n\left(m\right)&=\boxi_{m'+1}\Bigg(Q\bigg(z\Big(\Pi^{-1}_1\Big), \ldots, z\Big(\Pi^{-1}_{m'}\Big), z\Big(n'+\frac{1}{2}\Big)\bigg)\Bigg)
\\&=\boxi_{m'+1}\Bigg(Q\bigg(\frac{z\left(\Pi^{-1}_1\right)}{w}, \ldots, \frac{z\left(\Pi^{-1}_{m'}\right)}{w}, \frac{z\left(n'+\frac{1}{2}\right)}{w}\bigg)\Bigg)
\\&=\boxi_{m'+1}\Big(Q\big(Y_1, \ldots, Y_{m'}, A_n\big)\Big)
\\&=\square_{m'}(A_n).
\end{align*}

Let $$G_{n}(x):=\frac{\square_{m'}(x)}{\sqrt{Tnw}}.$$ \\
The function $\square_{m'}$ is increasing with respect to the relation $\prec$ (of~\cref{order}).
Hence $G_{n}$ also is increasing with respect to the relation $\prec$, i.e. if $x_1\leq x_2$, then
\begin{align}
G_{n}(x_1)\prec G_{n}(x_2).
\label{ineq}
\end{align}
From \cref{thm:romthm} for each $x\in[0,1]$ the random variable $G_n(x)$ converges in probability to the limit $G(x)$, when $n$ tends to infinity. Indeed
\begin{align*}
G_{n}(x)=\sqrt{\frac{m'}{n}\frac{1}{Tw}} \hspace{5pt} \frac{\square_{m'}(x)}{\sqrt{m'}}\overset{p}{\to} \sqrt{\frac{Tw}{Tw}}\hspace{5pt}G(x)=G(x).
\end{align*}
\end{proof}

\subsection{The uniform convergence on the interval}

In this section we prove the uniform convergence on the interval $[1, R]$. 
\begin{proof}
For each $x$, the sequence $G_n(x)$ converges in probability to the limit $G(x)$, 
when $n$ tends to infinity. Both coordinates of each function $G_n(x)$ are monotonic 
and the limit function $G(x)$ is continuous. Then from the Second Dini's Theorem 
we get that for each $x$, the sequence $G_n(x)$ converges uniformly in probability 
to the limit $G(x)$, when $n$ tends to infinity.

Since $$\left\|G_n(A_n)-G(A)\right\| \leq \left\|G_n(A_n)-G(A_n)\right\| + \left\|G(A_n)-G(A)\right\|,$$ 
the sequence $G_n(A_n)$ converges uniformly in probability to the limit $G(A)$, when $n$ tends to infinity. 
Then for each $A\in[1,R]$ and each $\epsilon>0$ we have:
$$\lim_{n\to\infty}\mathbb{P}\bigg(\sup_{A\in[\frac{1}{R},1]}\left\|G_{n}(A_n)-G(A)\right\| 
> \epsilon\bigg)=0.$$
Using the inequality $T\leq R$ we obtain

$$\lim_{n\to\infty}\mathbb{P}\bigg(\sup_{A\in[\frac{1}{R},1]}\sqrt{T}\left\|G_{n}(A_n)-G(A)\right\| 
> \sqrt{R}\epsilon\bigg)=0,$$
$$	\lim_{n\to\infty}\mathbb{P}\Bigg(\sup_{T\in[1,R]} \left\|\frac{\Pos_n\left(\floor{Tn}\right)}
{\sqrt{wn}}-H\left(T\right)\right\|>\sqrt{R}\epsilon\Bigg)=0.$$
Thus for each $\epsilon>0$ we have
$$	\lim_{n\to\infty}\mathbb{P}\Bigg(\sup_{T\in[1,R]} \left\|\frac{\Pos_n\left(\floor{Tn}\right)}
{\sqrt{wn}}-H\left(T\right)\right\|>\epsilon\Bigg)=0.$$
\end{proof}

\subsection{Further questions}
We see that the function $H$ behaves asymptotically for $T\to\infty$ like the function $(\frac{1}{\sqrt{T}}, 2\sqrt{T})$. The function $H$ describes the asymptotic position of the box with a fixed number $w$. The $x$-coordinate of this position after scaling tends to zero. If we did not scale it, would it also be similar? Will the box with the number $w$ be moved to the first column? We expect that almost surely yes. How long will we have to wait for this? We expect that if the number $w$ appears in the RSK algorithm after $n$ steps, the waiting time for the number in the first column is $O(n^2)$. Is this really true? More formally, let $\{X_j\}_{j=1}^\infty$ be a sequence of independent random variables with the uniform distribution $U(0,1)$. What is the probability that the box with a fixed number $w$ will be in the first column in the insertion tableau $P(x_1, \ldots, X_n, w, X_{n+1}, \ldots, X_{\floor{Tn^2}})$ depending on parameter $T$?

\section*{Acknowledgments}
I would like to thank my supervisor, Professor Piotr Śniady, for his support, the understanding, the patience, the positive outlook, the guidance, the valuable advice, the useful and constructive recommendations during writing this article. I would like to thank also to doctor Jacinta Torres for the help with the text editing.


\bibliography{bib}{}
\bibliographystyle{amsalpha}



\end{document}

