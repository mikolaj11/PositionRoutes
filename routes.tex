\documentclass[12pt]{article}

\usepackage[utf8]{inputenc}
\usepackage[english]{babel}
\usepackage{lmodern}
\usepackage{amsfonts, amsthm, amsmath}
\usepackage{breqn}

\newtheorem{theorem}{Theorem}
\DeclareMathOperator{\Pos}{Pos}
\DeclareMathOperator{\boxi}{box}
\newcommand{\floor}[1]{\lfloor #1 \rfloor}
\newcommand{\strzalka}{\hspace{-3pt}\uparrow}

\title{\textbf{Hydrodynamic limit of Robinson--Schensted-Knuth algorithm}}
\author{Miko\l{}aj Marciniak}
\date{}
\begin{document}

\maketitle

\begin{abstract}
We investigate evolution in the time the position of the fixed number in the insertion tableau, when the Robinson-Schensted-Knuth algorithm is applied to a finite sequence of random numbers. When the length of the sequence tends to infinity, a typical trajectory after scaling converges in probability to the some deterministic curve.
\end{abstract}

\section{Introduction}
A partition of a natural number $n$ is a representation of the number $n$ as the sum $n=\lambda_1+\lambda_2+\ldots+\lambda_k$, where $\lambda_1\geq \lambda_2\geq\ldots\geq\lambda_k>0$. The vector $\lambda=(\lambda_1,\lambda_2,\ldots,\lambda_k)$ is usually used to denote a partition. $\lambda-n$ denotes that $\lambda$ is the partition of number $n$. The Young diagram $\lambda$ is a graphical representation of the partition $\lambda$. The Young diagram $\lambda=(\lambda_1,\lambda_2,\ldots,\lambda_k)$ is a finite collection of boxes arranged in left-justified rows with the row lengths $\lambda_j$ in the $j-th$ row. A young tableau is a young diagram filled with numbers. If the entries strictly decrease along each column and weakly increase along each row, a tableau is called semistandard. A standard Young tableau is a semistandard Young tableau with $n$ boxes contains only the numbers $1, 2, \ldots, n$.  \\
\\\\\\ obrazek1 \\\\\\
Let $S_n$ denote the permutation group of order $n$. The Robinson-Schensted-Knuth correspondence RSK is a bijective correspondence that assigns to any permutation $\sigma\in S_n$ a pair of standard Young tableaux $(P,Q)$ with the same shape $\lambda-n$. $P$ is called an insertion tableau, and $Q$ is called a recording tableau. The RSK algorithm is based on multiple repeat of the insertion step to successive numbers from a given finite sequence. The insertion step takes as input the previously obtained tableau T and next number x from the sequence. Next produces as output a new tableau increased by one box. The insertion step consists in inserting the number x into first box containing number y greater than x. Move to the next row with the number y and repeat the action. At some row we are forced to insert the number at the end of the row. A collection of rearranged boxes is called the bumping route. \\\\
\\\\\\ obrazek2 \\\\\\
We can say that boxes with certain numbers was sliding by the bumping route. Each box in the Young tableau can be move over trough the bumping routes. Perhaps it is worth considering the position of the box with the selected number in time. 
\section{Motivations}
For many years, mathematicians have been studying the asymptotic behavior of the insertion tableau when we apply the Robinson-Schensted-Knuth algorithm to random input. \\\\
Logan and Shepp in \cite{} and independently Vershik and Kerov in \cite{} solved to the Ulam--Hammersley problem concerning typical lenght of a longest increasing subsequence in random permutations. The longest increasing subsequence also describes the length of first row in the insertion tableu obtained when we apply the RSK algorithm to the random finite sequence.\\\\
Dan Romik and Piotr Sniady in \cite{RS16} considered limit shape of bumping routes obtained from insertion tableau, when applying an RSK insertion step with a fixed number to an existing insertion tableau obtained from the random finite sequence. \\\\
In \cite{RS15} they considered the limit  shape  theorem  for  Plancherel-distributed  random  Young  diagrams and the limit shape of jeu de taquin obtained from insertion tableau made from the random finite sequence preceded by a fixed number. \\\\
Romik and Sniady considered two cases. In the first a fixed number was at the beginning of random sequence and in the second at the end. We can merge the cases considered by them and consider insertion tableau obtained by using an RSK algorithm applied to a random finite sequence containing a fixed number at some index. The box with fixed number can be sliding by the bumping routes. It is the very natural question about the position of the box with the number of x.\\\\
What can we say about the evolution over time of insertion tableau, when we apply the RSK algorithm to the sequence of independent and identically distributed (i.i.d.)$\hspace{3pt}U(0,1)$ random variables?
If investigate the position of a box with a fixed number, will we get a certain limit, when the number of boxes tends to infinity?
\section{The main result}
The main result of the article is a theorem describing asymptotic behavior of a fixed box. When the number of boxes tends to infinity then the position of the box with the fixed number converges in probability to the function G(A)=[tu bedzie definicja].\\

Let $w\in[0,1]$ be a fixed number. Let $\{X_j\}_{j=1}^\infty$ be a  sequence of i.i.d.$\hspace{3pt}U(0,1)$ random variables. For every $n\in\mathbb{N}$ let function $\Pos_n:\{n+1, n+2, \ldots\}\to\mathbb{N}^2$ describe coordinates of the box with number $w$ in the insertion tableau:
$$\Pos_n\left(j\right)=\boxi_w\left(P\left(X_1, \ldots, X_n, w, X_{n+1}, \ldots, X_j\right)\right)$$ for $j\in\{n+1, n+2, \ldots\}$. We consider the natural number $m\in \mathbb{N}$, which is approximately $A$ times greater than $n$, which means $\lim_{n\to\infty}\frac{m}{n}=A\in(1,\infty)$. For example, let $m$ be equal to $m=m(n)=\floor{An}$. 
\begin{theorem}
The random variable $\Pos_n(m)$, after scaling by $\sqrt{wn}$ converges in probability to the limit $G(A)$, when $n$ tends to infinity. Moreover the convergence is uniform $$\forall_{\epsilon>0} \lim_{n\to\infty}\mathbb{P}\left(\sup_{A\geq 1} \left\|G\left(A\right)-\frac{\Pos_n\left(m\right)}{\sqrt{wn}}\right\|>\epsilon\right)=0$$
\end{theorem}
\begin{proof}
We apply the RSK algorithm to a random sequence of real numbers containing the number $w$ and investigate the position of the box with the number $w$ in the insertion tableau. Numbers greater than $w$ do not affect the position of the box with the number $w$, so it is enough to consider only the subsequence containing numbers no greater than $w$.\\\\ 
Now we will give the formal proof.
The probability that the same number will occur twice in the sequence $(w, X_1, X_2, \ldots)$ is equal to 0, hence without losing generality, we assume that all numbers $w, X_1, X_2, \ldots$ are different. Let the sequence $\{X'_j\}_{j=1}^\infty$ be the subsequence of the sequence $\{X_j\}_{j=1}^\infty$ containing all elements of the sequence $\{X_j\}_{j=1}^\infty$, which are less than $w$. The sequence $\{X'_j\}_{j=1}^{\infty}$ is a sequence of i.i.d.$\hspace{3pt}U(0,w)$ random variables. Let $n'=n'(n)$ and $m'=m'(n)$ denote the numbers of element of sequences $\{X_j\}_{j=1}^n$, $\{X_j\}_{j=1}^m$ smaller than $w$. Then there is an equality:
\begin{dmath*}
\Pos_n\left(m\right)=\boxi_w\left(P\left(X_1, \ldots, X_n, w, X_{n+1}, \ldots, X_m\right)\right)=\boxi_w\left(P\left(X_1, \ldots, X'_{n'}, w, X'_{n'+1}, \ldots, X'_{m'}\right)\right).
\end{dmath*}
In addition the random variable $n'$ count how many numbers from the sequence $\{X_j\}_{j=1}^{n}$ are less than $w$, so $n'$ is the random variable with binomial distribution with parameters $n$ and $w$: $n'\sim B(n,w)$. Likewise the random variable $m'-n'$ count how many numbers from the sequence $\{X_j\}_{j=n'+1}^{m'}$ are less than $w$, so $m'-n'\sim B(m-n,w)$ 
Moreover, the random variables $n'$ and $m'-n'$ are independent, because the random variables $X_1, X_2, \ldots$ are independent. \\
From the Khinchin-Kolmogorov-Etemadi Strong Law of Large Numbers theorem, we know that if $n$ tends to infinity, then with probability $1$ exists the limits:
$$
\left\{
\begin{matrix}
\lim_{n\to\infty}\frac{n'}{n}&=&w, \\
\lim_{n\to\infty}\frac{m'-n'}{m-n}&=&w. \\
\end{matrix}
\right.
$$
Therefore, with probability 1 also exist the limit:
\begin{dmath*}
\lim_{n\to\infty}\frac{m'}{n'}=\lim_{n\to\infty}1+\frac{m'-n'}{n'}=1+\lim_{n\to\infty}\frac{m'-n'}{m-n}\frac{1}{\frac{n'}{n}}\frac{m-n}{n}=1+\lim_{n\to\infty}\frac{m'-n'}{m-n}\frac{1}{\frac{n'}{n}}\left(\frac{m}{n}-1\right)=1+\frac{w}{w}\left(A-1\right)=A.
\end{dmath*}

where the sequence $\{X'_j\}_{j=1}^\infty$ is a sequence of i.i.d. $\hspace{3pt}U(0,z)$.\\

We define the function $z:\{1, 2, \ldots m'\}\cup\{m', n'+\frac{1}{2}\}\to [0,1]$, that assigns to the number $t$, the $t-th$ largest number among $X'_1, X'_2, \ldots, X'_{m'}$ and additionally $z\left(m'+1\right)=w$ and $z\left(n'+\frac{1}{2}\right)=\frac{z\left(n'\right)+z\left(n'+1\right)}{2}$. 
\\

The sequence $\{z(j)\}_{j=1}^{m'}$ contains all elements of the sequence $\{X'_j\}_{j=1}^{m'}$ in the ascending order. In addition the sequence $\{X'_j\}_{j=1}^{m'}$ is a random permutation $\Pi$ with uniform distribution of the sequence $\{z(j)\}_{j=1}^{m'}$. The sequence $\{z(j)\}_{j=1}^{m'}$ will be called a random increasing sequence with a uniform distribution on the interval $[0,1]$. Let $\Pi=\left(\Pi_1, \Pi_2, \ldots, \Pi_{m'}\right)$. Then
$$\{X'_j\}_{j=1}^{m'}=\{z(\Pi_j)\}_{j=1}^{m'}=z\circ\Pi,$$
where $z$ is the function that acts separately on every element of the permutation $\Pi$. 
\\\\
Similarly the function $z$ act on Young tableau by acting on each box individually. Then
\begin{dmath*}
\Pos_n\left(m\right)=\boxi_w\left(P\left(X'_1, \ldots, X'_{n'}, w, X'_{n'+1}, \ldots, X'_{m'}\right)\right)
=\boxi_w\left(P\left(z(\Pi_1), \ldots, z\left(\Pi_{n'}\right), w, z\left(\Pi_{n'+1}\right), \ldots, z\left(\Pi_{m'}\right)\right)\right)
=\boxi_w\left(z\circ P\left(\Pi_1, \ldots, \Pi_{n'}, m'+1, \Pi_{n'+1}, \ldots, \Pi_{m'}\right)\right)
=\boxi_{m'+1}\left(P\left(\Pi_1, \ldots, \Pi_{n'}, m'+1, \Pi_{n'+1}, \ldots, \Pi_{m'}\right)\right).
\end{dmath*}

We define the permutation $\Pi\strzalka=\left(\Pi_1, \ldots, \Pi_{n'}, m'+1, \Pi_{n'+1}, \ldots, \Pi_{m'}\right)$ as a natural extension of the permutation $\Pi$. In additional we will use the fact that for any permutation $\Pi\strzalka$ the insertion tableau of $\Pi\strzalka$ is equal to the recording tableau of $\Pi\strzalka^{-1}$:
$$P\left(\Pi\strzalka\right)=Q\left(\Pi\strzalka^{-1}\right).$$
Therefore
\begin{dmath*}
\Pos_n\left(m\right)=\boxi_{m'+1}\left(P\left(\Pi_1, \ldots, \Pi_{n'}, m'+1, \Pi_{n'+1}, \ldots, \Pi_{m'}))=boxi_{m'+1}(P(\Pi\strzalka\right)\right)
=\boxi_{m'+1}\left(Q\left(\Pi\strzalka^{-1}\right)\right)
=\boxi_{m'+1}\left(Q\left(\Pi^{-1}_1\strzalka, \ldots, \Pi^{-1}_{m'}\strzalka, n'+1\right)\right).
\end{dmath*}
Now, using the function $z$, we will try to get the sequence of i.i.d.$\hspace{3pt}U(0,1)$ 
\begin{dmath*}
\Pos_n\left(m\right)=\boxi_{m'+1}\left(Q\left(\Pi^{-1}_1\strzalka, \ldots, \Pi^{-1}_{m'}\strzalka, n'+1\right)\right)\hspace{50pt}
=\boxi_{m'+1}\left(Q\left(\Pi^{-1}_1\strzalka, \ldots, \Pi^{-1}_{m'}\strzalka, n'+\frac{1}{2}\right)\right)
=\boxi_{m'+1}\left(Q\left(\Pi^{-1}_1, \ldots, \Pi^{-1}_{m'}, n'+\frac{1}{2}\right)\right)
=\boxi_{m'+1}\left(z\circ Q\left(\Pi^{-1}_1, \ldots, \Pi^{-1}_{m'}, n'+\frac{1}{2}\right)\right)
=\boxi_{m'+1}\left(Q\left(z\left(\Pi^{-1}_1\right), \ldots, z\left(\Pi^{-1}_{m'}\right), z\left(n'+\frac{1}{2}\right)\right)\right).
\end{dmath*}
$\Pi$ is the random permutation with uniform distribution, so $\Pi^{-1}$ is also random permutation with uniform distribution, then the sequence $z\circ\Pi^{-1}=\left(z\left(\Pi^{-1}_1\right), \ldots, z\left(\Pi^{-1}_{m'}\right)\right)$ is a sequence of i.i.d.$\hspace{3pt}U(0,w)$. We define the random variable $S_n$ and the sequence $\{Y_j\}_{j=1}^{m'}$. 
$$
\left\{
\begin{matrix}
Y_j=\frac{z\left(\Pi_j^{-1}\right)}{w} & \text{for}\hspace{10pt}j\in{1, 2, \ldots, m'}\\
S_n=\frac{z\left(n'+\frac{1}{2}\right)}{w}  
\end{matrix}
\right.
$$
The sequence $\{Y_j\}_{j=1}^{m'}$ is a sequence of i.i.d$\hspace{3pt}U(0,1)$. In additional random variable $S_n$ converges with probability $1$ to $\frac{1}{A}$:
\begin{dmath*}
\lim_{n\to\infty}S_n=\lim_{n\to\infty}\frac{z\left(n'+\frac{1}{2}\right)}{w}=\lim_{n\to\infty}\frac{1}{w}\frac{z\left(n'\right)+z\left(n'+1\right)}{2}=\lim_{n\to\infty}\frac{z\left(n'\right)}{w}+\lim_{n\to\infty}\frac{z\left(n'+1\right)-z\left(n'\right)}{2w}=\lim_{n\to\infty}\frac{z\left(n'\right)}{w}=\lim_{n\to\infty}\frac{n'}{m'}=\frac{1}{A}.
\end{dmath*}
Therefore
\begin{dmath*}
\Pos_n\left(m\right)=\boxi_{m'+1}\left(Q\left(z\left(\Pi^{-1}_1\right), \ldots, z\left(\Pi^{-1}_{m'}\right), z\left(n'+\frac{1}{2}\right)\right)\right)
=\boxi_{m'+1}\left(Q\left(\frac{z\left(\Pi^{-1}_1\right)}{w}, \ldots, \frac{z\left(\Pi^{-1}_{m'}\right)}{w}, \frac{z\left(n'+\frac{1}{2}\right)}{w}\right)\right)
=\boxi_{m'+1}\left(Q\left(Y_1, \ldots, Y_{m'}, S_n\right)\right).
\end{dmath*}
\end{proof}

\bibliography{bib}
\bibliographystyle{alpha}

\end{document}