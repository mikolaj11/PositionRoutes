\documentclass[12pt]{article}

\usepackage[utf8]{inputenc}
\usepackage[english]{babel}
\usepackage{lmodern}
\usepackage{amsfonts, amsthm, amsmath}
\usepackage{breqn}

\newtheorem{theorem}{Theorem}
\DeclareMathOperator{\Pos}{Pos}
\DeclareMathOperator{\boxi}{box}
\newcommand{\floor}[1]{\lfloor #1 \rfloor}
\newcommand{\strzalka}{\hspace{-3pt}\uparrow}

\title{\textbf{Hydrodynamic limit shape in the Young tableau}}
\author{Miko\l{}aj Marciniak}
\date{}
\begin{document}

\maketitle

\section{Introduction}
tu bedzie streszczenie i pewnie odwołania do \cite{RS16}\cite{RS15}.
\\
\\
\\
\section{Hydrodynamic limit shape}
\begin{theorem}
Let $w\in[0,1]$ be a fixed number. Let $\{X_j\}_{j=1}^\infty$ be a  sequence of i.i.d.$\hspace{3pt}U(0,1)$. Let $A\in(1,\infty)$, $n\in\mathbb{N}$ and let function $\Pos_n:\{n+1, n+2, \ldots\}\to\mathbb{N}^2$ describe position of box with number $z$:
$$Pos_n(m)=\boxi_w(P(X_1, \ldots, X_n, w, X_{n+1}, \ldots, X_j))$$ for $j\in\{n+1, n+2, \ldots\}$. We consider the natural number $m\in \mathbb{N}$, which is approximately $A$ times greater than $n$. For example, let $m$ be equal to $m=\floor{An}$. Then the $\Pos_n(m)$ function, after scaling by $\sqrt{w}\sqrt{n}$ goes in probability to a certain limit shape. Namely $$\exists_{G:[1,\infty)\to\mathbb{R}_+^2}\forall_{\epsilon>0} \lim_{n\to\infty}\mathbb{P}(\sup_{1\leq A} ||G(A)-\frac{Pos_n([An])}{\sqrt{w}\sqrt{n}}||>\epsilon)=0$$
\end{theorem}
\begin{proof}
The probability that the same number will occur twice in the sequence $(w, X_1, X_2, \ldots)$ is equal to 0. Without losing generality, we assume that all numbers $w, X_1, X_2, \ldots$ are different. Let the sequence $\{X'_j\}_{j=1}^\infty$ be a subsequence of the sequence $\{X_j\}_{j=1}^\infty$ containing all elements of the sequence $\{X_j\}_{j=1}^\infty$, which are less than $w$. The sequence $\{X'_j\}_{j=1}^{\infty}$ is a sequence of i.i.d$\hspace{3pt}U(0,w)$. Let $n'$ and $m'$ denote the numbers of element of sequences $\{X_j\}_{j=1}^n$, $\{X_j\}_{j=1}^m$ smaller than $w$. Namely
$$
\left\{
\begin{matrix}
n'=\#\{X_j | X_j<w,j\leq n\}, \\
m'=\#\{X_j | X_j<w,j\leq m\}=n'+\#\{X_j | X_j<w,n<j\leq m\}, 
\end{matrix}
\right.
$$
$$
\left\{
\begin{matrix}
n'=\#\{X_j | X_j<w,j\leq n\}\sim B(n,w), \\
m'-n'=\#\{X_j | X_j<w,n<j\leq m\}\sim B(m-n,w). 
\end{matrix}
\right.
$$
Moreover, the random variables $n'$ and $m'-n'$ are independent, because the random variables $X_1, X_2, \ldots$ are independent. \\
From the Khinchin-Kolmogorov-Etemadi Strong Law of Large, we known that numbers $\frac{n'}{n}$ and $\frac{m'-n'}{m-n}$ converges almost surely to $w$. With probability $1$ exists the limits:
$$
\left\{
\begin{matrix}
\lim_{n\to\infty}\frac{n'}{n}=w, \\
\lim_{n\to\infty}\frac{m'-n'}{m-n}=w. \\
\end{matrix}
\right.
$$
Therefore, with probability 1 also exist the limit:
\begin{dmath*}
\lim_{n\to\infty}\frac{m'}{n'}=\lim_{n\to\infty}1+\frac{m'-n'}{n'}=1+\lim_{n\to\infty}\frac{m'-n'}{m-n}\frac{1}{\frac{n'}{n}}\frac{m-n}{n}=1+\lim_{n\to\infty}\frac{m'-n'}{m-n}\frac{1}{\frac{n'}{n}}(\frac{\floor{An}}{n}-1)=1+\frac{w}{w}(A-1)=A.
\end{dmath*}
We have: 
\begin{dmath*}
\boxi_w(P(X_1, \ldots, X_n, z, X_{n+1}, \ldots, X_m))=\boxi_w(P(X_1, \ldots, X'_{n'}, z, X'_{n'+1}, \ldots, X'_{m'})),
\end{dmath*}
where the sequence $\{X'_j\}_{j=1}^\infty$ is a sequence of i.i.d. $\hspace{3pt}U(0,z)$.

We define the function $z:\{1, 2, \ldots m'\}\cup\{m', n'+\frac{1}{2}\}\to [0,1]$, that assigns to the number $t$, the $t-th$ largest number among $X'_1, X'_2, \ldots, X'_{m'}$.
$$
z(j)=
\left\{
\begin{matrix}
\min(\{X'_1, X'_2, \ldots X'_{m'}\}\setminus\{z(1), z(2), \ldots z(j-1)\}) & j\in\{1, 2, \ldots, m'\}\\
w & j=m'+1\\
\frac{z(n')+z(n'+1)}{2} & j=n'+\frac{1}{2}
\end{matrix}
\right.
$$
The sequence $\{z(j)\}_{j=1}^{m'}$ contains all elements of the sequence $\{X'_j\}_{j=1}^{m'}$ in the ascending order. The sequence $\{X'_j\}_{j=1}^{m'}$ is a random permutation $\Pi$ with uniform distribution of the sequence $\{z(j)\}_{j=1}^{m'}$. The sequence $\{z(j)\}_{j=1}^{m'}$ will be called a random increasing sequence with a uniform distribution on the interval $[0,1]$. Let $\Pi=(\Pi_1, \Pi_2, \ldots, \Pi_{m'})$. Then:
$$\{X'_j\}_{j=1}^{m'}=\{z(\Pi_j)\}_{j=1}^{m'}=z\circ\Pi,$$
where $\Pi$ is the function that acts separately on every element of the sequence $\{z(j)\}_{j=1}^{m'}$. 
\\\\
Similarly the function $z$ act on Young tableau by acting on each box individually. Then
\begin{dmath*}
\boxi_w(P(X'_1, \ldots, X'_{n'}, w, X'_{n'+1}, \ldots, X'_{m'}))
=\boxi_w(P(z(\Pi_1), \ldots, z(\Pi_{n'}), w, z(\Pi_{n'+1}), \ldots, z(\Pi_{m'})))
=\boxi_w(z\circ P(\Pi_1, \ldots, \Pi_{n'}, m'+1, \Pi_{n'+1}, \ldots, \Pi_{m'}))
=\boxi_{m'+1}(P(\Pi_1, \ldots, \Pi_{n'}, m'+1, \Pi_{n'+1}, \ldots, \Pi_{m'})).
\end{dmath*}

We define the permutation $\Pi\strzalka=(\Pi_1, \ldots, \Pi_{n'}, m'+1, \Pi_{n'+1}, \ldots, \Pi_{m'})$ as a natural extension of the permutation $\Pi$. In additional we will use the fact that for any permutation $\Pi\strzalka$ the insertion tableau of $\Pi\strzalka$ is equal to the recording tableau of $\Pi\strzalka^{-1}$:
$$P(\Pi\strzalka)=Q(\Pi\strzalka^{-1}).$$
Therefore:
\begin{dmath*}
\boxi_{m'+1}(P(\Pi_1, \ldots, \Pi_{n'}, m'+1, \Pi_{n'+1}, \ldots, \Pi_{m'}))=boxi_{m'+1}(P(\Pi\strzalka))
=\boxi_{m'+1}(Q(\Pi\strzalka^{-1}))
=\boxi_{m'+1}(Q(\Pi^{-1}_1\strzalka, \ldots, \Pi^{-1}_{m'}\strzalka, n'+1)).
\end{dmath*}
Now, using the function $z$, we will try to get the sequence of i.i.d.$\hspace{3pt}U(0,1)$ 
\begin{dmath*}
\boxi_{m'+1}(Q(\Pi^{-1}_1\strzalka, \ldots, \Pi^{-1}_{m'}\strzalka, n'+1))\hspace{50pt}
=\boxi_{m'+1}(Q(\Pi^{-1}_1\strzalka, \ldots, \Pi^{-1}_{m'}\strzalka, n'+\frac{1}{2}))
=\boxi_{m'+1}(Q(\Pi^{-1}_1, \ldots, \Pi^{-1}_{m'}, n'+\frac{1}{2}))
=\boxi_{m'+1}(z\circ Q(\Pi^{-1}_1, \ldots, \Pi^{-1}_{m'}, n'+\frac{1}{2}))
=\boxi_{m'+1}(Q(z(\Pi^{-1}_1), \ldots, z(\Pi^{-1}_{m'}), z(n'+\frac{1}{2}))).
\end{dmath*}
$\Pi$ is the random permutation with uniform distribution, so $\Pi^{-1}$ is also random permutation with uniform distribution, then the sequence $z\circ\Pi^{-1}=(z(\Pi^{-1}_1), \ldots, z(\Pi^{-1}_{m'}))$ is a sequence of i.i.d.$\hspace{3pt}U(0,w)$. We define the random variable $S_n$ and the sequence $\{Y_j\}_{j=1}^{m'}$. 
$$
\left\{
\begin{matrix}
Y_j=\frac{z(\Pi_j^{-1})}{w} & for\hspace{5pt}j\in{1, 2, \ldots, m'}\\
S_n=\frac{z(n'+\frac{1}{2})}{w}  
\end{matrix}
\right.
$$
The sequence $\{Y_j\}_{j=1}^{m'}$ is a sequence of i.i.d$\hspace{3pt}U(0,1)$. In additional random variable $S_n$ converges almost surely to $\frac{1}{Aw}$:
\begin{dmath*}
\lim_{n\to\infty}S_n=\lim_{n\to\infty}\frac{z(n'+\frac{1}{2})}{w}=\lim_{n\to\infty}\frac{1}{w}\frac{z(n')+z(n'+1)}{2}=\lim_{n\to\infty}\frac{z(n')}{w}+\lim_{n\to\infty}\frac{z(n'+1)-z(n')}{2w}=\frac{1}{w}\lim_{n\to\infty}z(n')=\frac{1}{w}\lim_{n\to\infty}\frac{n'}{m'}=\frac{1}{Aw}.
\end{dmath*}
Therefore
\begin{dmath*}
\boxi_{m'+1}(Q(z(\Pi^{-1}_1), \ldots, z(\Pi^{-1}_{m'}), z(n'+\frac{1}{2})))
=\boxi_{m'+1}(Q(\frac{z(\Pi^{-1}_1)}{w}, \ldots, \frac{z(\Pi^{-1}_{m'})}{w}, \frac{z(n'+\frac{1}{2})}{w})
=\boxi_{m'+1}(Q(Y_1, \ldots, Y_{m'}, S_n)).
\end{dmath*}
\end{proof}

\bibliography{bib}
\bibliographystyle{alpha}

\end{document}