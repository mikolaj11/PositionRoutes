\documentclass[12pt]{article}

\usepackage[utf8]{inputenc}
\usepackage[english]{babel}
\usepackage{lmodern}
\usepackage{amsfonts, amsthm, amsmath}
\usepackage{breqn}
\newtheorem{theorem}{Theorem}

\title{\textbf{Limit shapes of position routes}}
\author{Mikolaj Marciniak and Piotr Sniady}
\date{}
\begin{document}

\maketitle

\section{Introduction}

Dan Romik and Piotr Sniady in \cite{RomikSniady1} were considering limit shape of bumping routes obtained from insertion tableau, when applying an RSK insertion step with a fixed input $z \in [0, 1]$ to an existing insertion tableau $P_n$, where $P_n$ is the result of $n$ previous insertion steps applying to $n$ i.i.d. $U(0, 1)$ random inputs $X_1, \ldots, X_n$.
\\\\
In \cite{RomikSniady2} they were considering limit shape of jeu de taquin path obtained from insertion tableau made from $n$ i.i.d. $U(0, 1)$ random inputs $X_1, \ldots, X_n$ prepending a fixed number $z \in [0, 1]$.
\\\\
They were looking into in the first case to $P((X_1, \ldots, X_n, z))$ and in the second case to $P((z, X_1, \ldots, X_n))$. We can merge this cases and consider $P((X_1, \ldots, X_n, z, X_{n+1}, \ldots, X_m))$, where $m$ is approximately $A$ times greater than $n$ for any $A\in(1,\infty]$.
\\\\
After applying an RSK insertion steps consecutive for numbers $X_1, \ldots, X_n, z$ in insertion tableau will appear box with number $z$. During applying an RSK insertion steps consecutive for numbers $X_{n+1}, \ldots, X_m$ box with number $z$ can be sliding by the bumping roots. We would like to prove, that exist macroscopic limit shape of path describing the position of box with number $z$.

\newpage

\section{Position path}
\begin{theorem}
Let's take a fixed number $z\in[0,1]$. Let $\{X_j\}_{j=1}^\infty$ will be a  sequence of i.i.d. $U(0,1)$. Let $A\in(1,\infty)$, $n\in\mathbb{N}$ and let function $Pos_n:\{n+1, n+2, \ldots\}\to\mathbb{N}^2$ describe position of box with number $z$:
$$Pos_n(j)=(a_j^{(n)},b_j^{(n)})=box_z(P(X_1, \ldots, X_n, z, X_{n+1}, \ldots, X_j))$$ for $j\in\{n+1, n+2, \ldots\}$. $\exists_{G:[1,\infty)\to\mathbb{R}_+^2}\forall_{\epsilon>0}$ 
\\
\center{$lim_{n\to\infty}\mathbb{P}(sup_{1\leq A} ||G(A)-\frac{Pos_n([An])}{\sqrt{A}\sqrt{n}}||>\epsilon)=0$} 
\end{theorem}
\begin{proof}
The probability that in our sequence there will be a repetition of numbers or numbers $z$ is equal to 0. Without losing generality, we assume that all numbers $z, X_1, X_2, \ldots$ are different. Numbers greater than $z$ do not affect to the position of the box with the number $z$, so we are only considering elements of the sequence $\{X_j\}_{j=1}^\infty$ of less than $z$. Let the sequence $\{X'_j\}_{j=1}^\infty$ be a subsequence of the sequence $\{X_j\}_{j=1}^\infty$ containing all elements of the sequence $\{X_j\}_{j=1}^\infty$, which are less than $z$. Let $m=[An]$. We have: $$box_z(P(X_1, \ldots, X_n, z, X_{n+1}, \ldots, X_m))=box_z(P(X_1, \ldots, X'_{n'}, z, X'_{n'+1}, \ldots, X'_{m'}))$$
where the sequence $\{X'_j\}_{j=1}^\infty$ is a sequence of i.i.d. $U(0,z)$ and
 
$$
\left\{
\begin{matrix}
\lim_{n\to \infty}\frac{m}{n}=\lim_{n\to \infty}\frac{[An]}{n}=A\\
n'=\#\{X_j | X_j<z,j\leq n\}=Binom(n,z) \\
m'=\#\{X_j | X_j<z,j\leq m\}=Binom(m,z) 
\end{matrix}
\right.
$$
Moreover, similarly as in [] we consider the sequence $\{z(j)\}_{j=1}^{m'}$, which is a sequence containing all elements of the sequence $\{X'_j\}_{j=1}^{m'}$ in ascending order. The sequence $\{z(j)\}_{j=1}^{m'}$ is random increasing sequence with a uniform distribution $U(0,z)$ and random variables $X'_1, \ldots, x'_{m'}$ have a uniform distribution, therefore the sequence $\{z(j)\}_{j=1}^{m'}$ is a random permutation $\Pi$ of sequence $\{X'_j\}_{j=1}^{m'}$, where random variables $\Pi$ is permutation of numbers $\{1, 2, \ldots, m'\}$ and also have a uniform distribution. For each $j\leq m'$, we have $X'_j=z(\Pi(j))$. For write legibility we put 

$$
\left\{
\begin{matrix}
z(m'+1)=z \\ 
z(m'+\frac{1}{2})=\frac{z(n')+z(n'+1)}{2}\\
zS_n=z(n'+\frac{1}{2})=\frac{z(n')+z(n'+1)}{2}
\end{matrix}
\right.
$$
Let $\Pi_0=(\Pi(1), \ldots, \Pi(n'), m'+1)$ and \\

$$
\Pi^{-1}(j)\uparrow = \left\{
\begin{matrix}
\Pi^{-1}(j) & j\leq n'\\
\Pi^{-1}(j)+1 & n'<j\leq m'\\
n'+1 & j=m'+1
\end{matrix}
\right.
$$
We will use the fact stated in [] that for any permutation $\Pi_0$ the insertion tableau of $\Pi_0$ is equal to the recording tableau of $\Pi_0^{-1}$:
$$P(\Pi_0)=Q(\Pi_0^{-1})$$
Therefore:
\begin{dmath*}
Pos_n(m)
=box_z(P(X_1, \ldots, X_n, z, X_{n+1}, \ldots, X_m))
=box_z(P(X'_1, \ldots, X'_{n'}, z, X'_{n'+1}, \ldots, X'_{m'}))
=box_z(P(z(\Pi(1)), \ldots, z(\Pi(n')), z, z(\Pi(n'+1)), \ldots, z(\Pi(m'))))
=box_z(z\circ P(\Pi(1), \ldots, \Pi(n'), m'+1, \Pi(n'+1), \ldots, \Pi(m')))
=box_{m'+1}(P(\Pi(1), \ldots, \Pi(n'), m'+1, \Pi(n'+1), \ldots, \Pi(m')))
=box_{m'+1}(P(\Pi_0))
=box_{m'+1}(Q(\Pi_0^{-1}))
=box_{m'+1}(Q(\Pi^{-1}(1)\uparrow, \ldots, \Pi^{-1}(m')\uparrow, n'+1))
=box_{m'+1}(Q(\Pi^{-1}(1)\uparrow, \ldots, \Pi^{-1}(m')\uparrow, n'+\frac{1}{2}))
=box(m'+1)(z\circ Q(\Pi^{-1}(1)\uparrow, \ldots, \Pi^{-1}(m')\uparrow, n'+\frac{1}{2}))
=box_{m'+1}(Q(zY_1, \ldots, zY_{m'}, zS_n))
=box_{m'+1}(Q(Y_1, \ldots, Y_{m'}, S_n))
\end{dmath*}
since the sequence $\{zY_j\}_{j=1}^{m'}$ is the random increasing sequence with a uniform distribution permutated by the random permutation with a uniform distribution. We receive that the sequence $\{Y_j\}_{j=1}^{m'}$ is a sequence of i.i.d. $U(0,1)$, $m'=Binom([An],z)$ and with probability 1 we have:
\begin{dmath*}
\lim_{n \to \infty} S_n=\lim_{n \to \infty} \frac{n'}{m'}=\lim_{n \to \infty} \frac{\frac{Binom(n,z)}{n}}{\frac{Binom(m,z)}{m}}\frac{n}{m}=\frac{1}{A}
\end{dmath*}
\end{proof}
\newpage
\begin{thebibliography}{9}
\bibitem{RomikSniady1}
Dan Romik, Piotr Śniady
\textit{Limit shapes of bumping routes in the Robinson-Schensted correspondence}.
\bibitem{RomikSniady2} 
Dan Romik, Piotr Śniady
\textit{Jeu de taquin dynamics on infinite Young tableaux and second class particles}. 
The Annals of Probability
2015, Vol. 43, No. 2: 719-724.
\end{thebibliography}


\end{document}
